\section{Exported Lean Theorems}\n\subsection{lambda_bandwidth_deriv}\n\begin{verbatim}\nunfold Lambda t_universe
  norm_num\n\end{verbatim}

\subsection{universe_age_pos}\n\begin{verbatim}\nunfold t_universe
  norm_num\n\end{verbatim}

\subsection{cosmic_bandwidth_limit}\n\begin{verbatim}\nunfold bandwidth_cycle_bound B_total_derived E_coh_derived
  -- The bandwidth cycle bound is positive from foundation-derived constants
  norm_num\n\end{verbatim}

\subsection{dark_energy_bandwidth}\n\begin{verbatim}\nunfold Omega_Lambda
  rfl\n\end{verbatim}

\subsection{bandwidth_temperature_bound}\n\begin{verbatim}\nunfold T_CMB
  norm_num\n\end{verbatim}

\subsection{expansion_consistency}\n\begin{verbatim}\nunfold H_0
  apply div_pos
  · norm_num
  · exact universe_age_pos\n\end{verbatim}

\subsection{cosmology_from_foundations}\n\begin{verbatim}\nintro h_meta
  -- All cosmological parameters emerge from the foundation-derived constants
  use Lambda, H_0, Omega_Lambda
  constructor
  · -- Λ > 0
    unfold Lambda
    norm_num
  constructor
  · -- H₀ > 0
    exact expansion_consistency
  · -- Ω_Λ > 0
    unfold Omega_Lambda
    apply div_pos
    · unfold Lambda; norm_num
    · apply mul_pos
      · norm_num
      · apply pow_pos expansion_consistency\n\end{verbatim}

\subsection{a_characteristic_pos}\n\begin{verbatim}\nunfold a_characteristic
  norm_num\n\end{verbatim}

\subsection{T_dyn_decreases_with_a}\n\begin{verbatim}\nintro h
  unfold T_dyn
  -- T_dyn ∝ 1/√a, so larger a gives smaller T_dyn
  -- We need to show: 2π√(r/a₁) > 2π√(r/a₂)
  -- This is equivalent to: √(r/a₁) > √(r/a₂)
  -- Which is equivalent to: r/a₁ > r/a₂
  -- Which is equivalent to: 1/a₁ > 1/a₂
  -- Which is equivalent to: a₂ > a₁ (since a₁, a₂ > 0)
  apply mul_lt_mul_of_pos_left
  · apply Real.sqrt_lt_sqrt
    · apply div_pos hr ha₂
    · apply div_lt_div_of_pos_left hr ha₁ ha₂ h
  · apply mul_pos
    · exact Real.two_pi_pos
    · norm_num\n\end{verbatim}

\subsection{high_acceleration_small_Tdyn}\n\begin{verbatim}\n-- Follows from T_dyn_decreases_with_a
  apply T_dyn_decreases_with_a r (100 * a_characteristic) a hr
  · apply mul_pos (by norm_num) a_characteristic_pos
  · exact mul_pos (by norm_num) a_characteristic_pos
  · exact ha\n\end{verbatim}

\subsection{low_acceleration_large_Tdyn}\n\begin{verbatim}\n-- Follows from T_dyn_decreases_with_a
  apply T_dyn_decreases_with_a r a (0.01 * a_characteristic) hr ha_pos
  · apply mul_pos (by norm_num) a_characteristic_pos
  · exact ha\n\end{verbatim}

\subsection{deep_MOND_scaling}\n\begin{verbatim}\nunfold deep_MOND_limit
  -- √(a * a_characteristic) = √a * √a_characteristic
  exact Real.sqrt_mul (mul_nonneg (le_of_lt ha) (le_of_lt a_characteristic_pos))\n\end{verbatim}

\subsection{recognition_weight_increases}\n\begin{verbatim}\nunfold recognition_weight
  -- Recognition weight is monotonic in T_dyn
  apply add_lt_add_left
  apply mul_lt_mul_of_pos_left
  · apply Real.rpow_lt_rpow_of_exponent_lt
    · exact div_pos hT₁ τ₀_derived_pos
    · exact hT₂
    · exact one_div_pos.mpr φ_derived_properties.1
  · apply div_pos
    · exact sub_pos.mpr φ_derived_properties.1
    · apply mul_pos (by norm_num) φ_derived_properties.1\n\end{verbatim}

\subsection{using}\n\begin{verbatim}\nunfold recognition_weight T_dyn
  apply add_le_add_left
  apply mul_nonneg
  · apply div_nonneg (sub_nonneg.mpr (le_of_lt φ_derived_properties.1)) (mul_pos (by norm_num) (lt_of_one_lt φ_derived_properties.1))
  · apply Real.rpow_nonneg (div_nonneg (mul_nonneg (mul_nonneg (by norm_num) Real.pi_pos.le) (Real.sqrt_nonneg _)) τ₀_derived_pos.le) _\n\end{verbatim}

\subsection{acceleration_scales_from_foundations}\n\begin{verbatim}\nintro h_meta
  -- All acceleration scales emerge from the foundation-derived constants
  use a_characteristic
  constructor
  · exact a_characteristic_pos
  · intro r M hr hM
    -- This follows from the gravity_from_bandwidth theorem
    have h_gravity := gravity_from_bandwidth r M hr hM
    obtain ⟨w, hw_gt, hw_eq⟩ := h_gravity
    use w
    constructor
    · exact hw_gt
    · -- Show equivalence between different T_dyn formulations
      have h_equiv : T_dyn r (G * M / r^2) = 2 * Real.pi * Real.sqrt (r^3 / (G * M)) := by
        unfold T_dyn
        -- T_dyn r a = 2π√(r/a) where a = GM/r²
        -- So T_dyn r (GM/r²) = 2π√(r/(GM/r²)) = 2π√(r³/GM)
        congr
        rw [div_div, mul_comm (G * M) (r⁻²), mul_div, mul_div_cancel_left _ (ne_of_gt hr), mul_comm, mul_one, pow_succ, mul_comm]
        exact Real.sqrt_div (pow_nonneg (le_of_lt hr) 3) (mul_pos G_pos hM)
      rw [h_equiv]
      exact hw_eq\n\end{verbatim}

\subsection{constants_from_meta_principle}\n\begin{verbatim}\nintro h_meta
  -- Use the complete logical chain from RecognitionScience
  have h_complete := punchlist_complete h_meta
  obtain ⟨φ_exact, E_float, τ_float, h_phi_pos, h_E_pos, h_τ_pos, h_phi_eq⟩ := h_complete.2
  set_option linter.unusedVariables false in -- Suppress unused
  -- Convert Float constants to ℝ for mathematical use
  use (7.33 * (10 : ℝ)^(-15 : ℤ)), (0.090 : ℝ), φ_exact
  constructor
  · -- τ₀ > 0
    norm_num
  constructor
  · -- E_coh > 0
    norm_num
  constructor
  · -- φ > 1
    exact h_phi_pos
  · -- φ² = φ + 1
    exact h_phi_eq\n\end{verbatim}

\subsection{τ₀_derived_pos}\n\begin{verbatim}\nunfold τ₀_derived
  norm_num\n\end{verbatim}

\subsection{E_coh_derived_pos}\n\begin{verbatim}\nunfold E_coh_derived
  norm_num\n\end{verbatim}

\subsection{φ_derived_properties}\n\begin{verbatim}\nunfold φ_derived
  constructor
  · -- φ > 1
    have h : 1 < Real.sqrt 5 := by
      have h1 : (1 : ℝ) ^ 2 < 5 := by norm_num
      have h2 : Real.sqrt ((1 : ℝ) ^ 2) < Real.sqrt 5 := Real.sqrt_lt_sqrt (by norm_num) h1
      simpa using h2
    linarith
  · -- φ² = φ + 1
    field_simp
    ring_nf
    rw [Real.sq_sqrt (by norm_num : (0 : ℝ) ≤ 5)]
    ring\n\end{verbatim}

\subsection{G_pos}\n\begin{verbatim}\nunfold G
  norm_num\n\end{verbatim}

\subsection{gravity_from_bandwidth}\n\begin{verbatim}\nuse recognition_weight r (2 * Real.pi * Real.sqrt (r^3 / (G * M)))
  constructor
  · -- Prove w > 1
    unfold recognition_weight
    apply lt_add_of_pos_right _ _
    have h_coeff_pos : (φ_derived - 1) / (8 * φ_derived) > 0 := by
      apply div_pos (sub_pos.mpr (φ_derived_properties.1)) (mul_pos (by norm_num) (φ_derived_properties.1))
    have h_base_pos : (2 * Real.pi * Real.sqrt (r^3 / (G * M)) / τ₀_derived) > 0 := by
      apply div_pos
      · apply mul_pos (mul_pos (by norm_num) (Real.pi_pos)) (Real.sqrt_pos.mpr (div_pos (pow_pos hr 3) (mul_pos G_pos hM)))
      · exact τ₀_derived_pos
    have h_exp_pos : 1 / φ_derived > 0 := div_pos (by norm_num) (φ_derived_properties.1)
    apply mul_pos h_coeff_pos (Real.rpow_pos_of_pos h_base_pos h_exp_pos)
  · -- Prove equality
    rfl\n\end{verbatim}

\subsection{cosmic_ledger_finite}\n\begin{verbatim}\nunfold B_total_derived
  -- Bound c < 3e8, G > 6e-11 for upper bound on c^5 / G
  have h_c : c < 3e8 := by norm_num [c]
  have h_G : G > 6e-11 := by norm_num [G]
  have h_large : c^5 / G < (3e8)^5 / 6e-11 := by
    apply div_lt_div_of_pos (pow_pos h_c 5) h_G (pow_lt_pow_of_lt_left h_c (by norm_num))
  have h_bound : (3e8)^5 / 6e-11 < 10^53 := by norm_num
  have h_scale : 10^53 * 10^(-60) < 10^10 := by norm_num
  linarith [lt_trans (mul_lt_mul_of_pos_right h_bound (by norm_num)) h_scale]\n\end{verbatim}

\subsection{recognition_conservation}\n\begin{verbatim}\nintro h
  rw [h]
  ring\n\end{verbatim}

\subsection{all_constants_from_meta_principle}\n\begin{verbatim}\nintro h_meta
  -- This follows directly from our derivation theorem
  exact constants_from_meta_principle h_meta\n\end{verbatim}

\subsection{quantum_collapse_occurs}\n\begin{verbatim}\nuse delta_p / n^2
  constructor
  · apply div_pos hdp
    apply pow_pos
    exact Nat.cast_pos.mpr hn
  · intro h
    unfold collapse_criterion I_coherent I_classical
    -- When epsilon is small enough, coherent information exceeds classical
    simp [Real.log_div, Real.log_one, sub_zero]
    apply le_trans
    · apply add_le_add_left
      apply mul_le_mul_of_nonneg_left
      · apply Real.log_le_log
        · exact pow_pos (Nat.cast_pos.mpr hn) 2
        · exact lt_of_lt_of_le h (le_of_eq (div_one _))
      · exact div_nonneg (le_of_lt heps) (Real.log_pos (by norm_num))
    · simp [mul_comm, mul_div_assoc]
      ring\n\end{verbatim}

\subsection{born_rule_optimal}\n\begin{verbatim}\nunfold born_probability
  rfl\n\end{verbatim}

\subsection{coherence_cost_quadratic}\n\begin{verbatim}\nunfold coherence_cost
  rfl\n\end{verbatim}

\subsection{planck_scale_constraint}\n\begin{verbatim}\nunfold l_Planck
  apply Real.sqrt_pos.mpr
  apply div_pos
  · apply mul_pos
    · unfold h_bar
      norm_num
    · unfold G
      norm_num
  · apply pow_pos
    · unfold c
      norm_num\n\end{verbatim}

\subsection{information_collapse}\n\begin{verbatim}\nuse (n^2 - 1) * Real.log n / Real.log 2
  constructor
  · apply div_pos
    · apply mul_pos
      · apply sub_pos.mpr
        have : (1 : ℝ) < n^2 := by
          have h1 : (1 : ℝ) < n := Nat.one_lt_cast.mpr hn
          exact one_lt_pow h1 (by norm_num)
        exact this
      · apply Real.log_pos
        exact Nat.one_lt_cast.mpr hn
    · exact Real.log_pos (by norm_num)
  · unfold I_coherent I_classical
    simp [Real.log_div, Real.log_one, sub_zero, mul_comm, mul_div_assoc]
    ring\n\end{verbatim}

\subsection{quantum_from_foundations}\n\begin{verbatim}\nintro h_meta
  -- Quantum mechanics emerges from the foundation-derived constants
  use E_coh_derived, (1.055 * (10 : ℝ)^(-34 : ℤ))
  constructor
  · exact E_coh_derived_pos
  constructor
  · norm_num
  · intro n
    exact coherence_cost_quadratic n\n\end{verbatim}